%%%%%%%%%%%%%%%%%%%%%%%%%%%%%%%%%%%%%%%%%%%%%%%%%%%
%
%  New template code for TAMU Theses and Dissertations starting Fall 2016.  
%
%
%  Author: Sean Zachary Roberson
%  Version 3.17.09
%  Last Updated: 9/21/2017
%
%%%%%%%%%%%%%%%%%%%%%%%%%%%%%%%%%%%%%%%%%%%%%%%%%%%

%%%%%%%%%%%%%%%%%%%%%%%%%%%%%%%%%%%%%%%%%%%%%%%%%%%%%%%%%%%%%%%%%%%%%%
%%                           SECTION I
%%%%%%%%%%%%%%%%%%%%%%%%%%%%%%%%%%%%%%%%%%%%%%%%%%%%%%%%%%%%%%%%%%%%%


\pagestyle{plain} % No headers, just page numbers
\pagenumbering{arabic} % Arabic numerals
\setcounter{page}{1}

\chapter{\uppercase {Introduction}}
The Finite Element Method (FEM)  is  a numerical technique that has been widely applied to solve partial differential equations (PDEs) arising from different types of boundary value problems such as heat transfer \cite{Reddy2010}, fracture mechanics \cite{Kuna2013}, earthquake rupture \cite{Duan2006}, mechanical deformation \cite{Lewis1998}, fluid flow  \cite{Hughes1986, Aarnes2008},  mass transport \cite{Sudicky1989}, seismic wave propagation \cite{Ham2012, Gao2015},  among others. 

The FEM  is a versatile numerical method that present several advantages.  It allows, in a straightforward manner, to incorporate flexible meshing techniques that conform to complex structures within the model domain \cite{DeBasabe2009, Frehner2008}, increasing the accuracy of the solution. The FEM also allows to impose easily natural boundary conditions through its weak formulation \cite{Brenner2008}. From a mathematical point of view, the FEM weak formulation makes it possible to prove the  uniqueness of its solution \cite{Brenner2008}. 

\section{FEM Applied to Wave Simulation}
The classical  continuous Galerkin FEM  with piecewise polynomial approximation has been applied for the simulation of acoustic  and seismic wave propagation \cite{Marfurt1984, Mullen1982}.  However,  one of the main simulation issues is the dispersion effect that the solution suffers as the wave number increases  \cite{Deraemaeker1999, Ihlenburg1995a}. Dispersion error  refers to the wave number difference between the numerical and exact solution \cite{Deraemaeker1999}, and depends on the spatial and temporal discretization of the numerical problem \cite{DeBasabe2007}.  The  simplest  way to improve the accuracy of the solution is to incur in increasingly  refined meshes \cite{Ihlenburg1995},  but  this approach  becomes  computational expensive  as the  wave number increases. Improved approaches include the  implementation of higher order polynomial approximation \cite{Esterhazy2017, Ihlenburg1997}, while modern techniques incorporate adaptivity of mesh refinement and of high order polynomials based on a posteriori error estimation \cite{Bangerth2009, Demkowicz1989}.

The generalized finite element method (GFEM) is a different FEM implementation strategy to improve the accuracy and efficiency of wave simulation. The GFEM applies the  partition of unity property of the standard FEM basis functions \cite{BABUSKA1997}. This approach relies on adding enrichment or user-defined basis functions,  apart form the standard polynomials, to enhance the solution approximation and  avoid excessive mesh refinement as the wave number increases. In general, the criterion to chose  additional basis functions is based on closed form solutions of particular partial differential equations \cite{Strouboulis2000, Babuska1997a}. This technique has been mostly applied to solve the  harmonic wave equation with a variety of oscillatory enrichment functions. In \cite{Strouboulis2006, Babuska1997a}, the authors propose plane waves at different directions as  additional enrichment functions to solve the Helmholtz equation,  showing the higher convergence rate of the solution compared to the standard FEM. In \cite{Strouboulis2008}, the authors  considered, apart from  plane waves enrichment functions,  wave band and Vekua functions, testing performance, convergence rate and meshes with different architectures. In \cite{ElKacimi2009}, the authors propose a solution for the time-harmonic elastic wave equation incorporating plane waves at different directions to enrich both compressional (P) and shear (S) waves. They show that it is possible to increase frequency without further mesh refinement while maintaining  the accuracy of the solution. As discussed, most of the problems treated in the literature that incorporate the GFEM approach are time harmonic. Although in \cite{Ham2012},  the authors implement transient problems, they test cases considering homogeneous media only. The GFEM as discussed falls in the continuous Galerkin (CG) formulation. However, as shown in \cite{Hiptmair2016}, a discontinuous Galrking (DG) formulation is also possible. In this formulation the continuity of the basis functions is not required at the DOF nodes, providing more flexibility in defining basis functions. However this method increases the number of DOFs for which to solve. Nevertheless, DG methods have the advantage of yielding block diagonal matrices that are more amenable to invert; which is generally not the case in CG formulations. For this thesis work I focus on the GFEM approach based on the CG formulation.

Similar methods to the GFEM, such as some versions of the generalized multiscale approach (GMsFEM) \cite{Gao2015,Jiang2010}, also rely on the partition of unity property to simulate wave propagation. The central aspect of these methods is the numerical estimation of enrichment functions by solving spectral local problems in a fine mesh with the goal of capturing local heterogenities. These enrichments are then incorporated as part of the basis functions to solve the time dependant problem in a coarser mesh. The main difference  of these multiscale methods with GFEM  is that GFEM aims for capturing the wave number, which depends on both  the medium seismic velocities and the induced frequency of an external source. In this sense, multiscale and the GFEM approaches can be seen as complementary, with one being able to  incorporate small scale heterogeneities and with the other one targeting to include the frequency signature of an external source.


\section{Challenges of  Wave Simulation in Exploration Seismology}
Important applications of wave simulation  in exploration seismology involve the  propagation  of wavefields across irregular boundaries found in the near-surface geology and  in surface topography \cite{Yilmaz2013, Keho2012, Bridle2007}. These type of features present challenges for the implementation of  meshing techniques to conform to the irregular boundaries and  for handling the high impedance contrast between these structures and the surrounding rock, which can lead to excessive  mesh refinement.  For instance, carbonate reservoirs  present  near-surface diagenetic features  that  are a product of massive dissolution, collapse and fracturing of rocks  \cite{Lucia1999a,Wright1994},  which  result in  the formation of caves, vugs  and fracture systems  with irregular geometries \cite{Huang2017, Robert2006}, adding complexity to the underground structures. These diagenetic products could be partially filled with different material such as loose sediments, breccias  or  water \cite{Regone2017, Lucia1999a }, which can  create a  high impedance contrast with the surrounding rock. Similarly, topography also imposes challenges on wave  simulation. Surface relief includes irregular structures such as sand dunes, dry river beds, salt flasts, collapse filled karsts  among others \cite{Keho2012, Bridle2007},  which in general  generate  wave scattering, surface ground rolls and may also trap seismic energy producing unwanted multiples \cite{Keho2012}. Thus, to  improve the accuracy of wave simulation  through these  complex features is paramount to model as accurate as possible  their  geometry and seismic properties. Finite difference (FD) techniques have been  widely used to simulate wave propagation since they have a faster run time than FEM. However, its main disadvantage is their lack of flexibility to mesh complex shapes. Although recent implementations have tried to improve on this issue, these techniques are  still less direct than FEM approaches \cite{Lan2011,Tessmer1992}. In contrast to FD,  FEM-related methods allow a straightforward treatment of irregular geometries as they can incorporate flexible,  boundary-conforming meshes \cite{Komatitsch1998, Lee2008}. Furthermore, the GFEM has the potential to improve the efficiency of the simulation time since this technique does not require incremental mesh refinement as the wave number increases, reducing largely the computational time of the FEM, which has been traditionally the most impactful disadvantage of the method. 

\section{Summary of the Thesis Work}
% redo this
For the present thesis work, I implement the GFEM approach to simulate the acoustic wave propagation by introducing plane waves at different directions with their wave number matching that of the geological feature with the highest wavenumber in a seismic model. These plane waves are introduced as the enrichment functions for the extended GFEM basis functions as in \cite{Strouboulis2006} to improve the efficiency of the solution convergence. However in \cite{Strouboulis2006} and similar work  \cite{Strouboulis2008, ElKacimi2009} the GFEM  implementation considered mainly the solution of standing waves. Thus, I expand its application to the transient wave propagation with particular focus on acoustic waves. Although, I  include an initial basic example of wave propagation on a homogeneous acoustic medium to show the main advantages of the GFEM method, I also present  examples with more relevance to exploration seismology. In special, I  consider media with complex underground structures that include a low velocity layer, a karst inclusion, and surface topography. For these cases, I use flexible meshing, capable to conform to the boundary of complicated geometries. I also present performance comparisons between the GFEM and a standard FEM reference solution, including estimation of the solution error respect to the reference and comparison of simulation times.


